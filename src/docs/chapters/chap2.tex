\chapter{结论}

评论家对游戏的养成系统和收费方式手段意见不一。《PC Gamer》的科尔普认为其收费方式“不算过分”,但其各种货币和奖励机制的复杂程度破坏了原本流畅的游戏体验。Screen Rant的格拉韦尔将该游戏作为一款抽卡游戏视为其唯一的缺陷,批评其五星角色的掉落率低,武器系统“光锥”更是让获得最佳角色配置的成本相当高,其体力系统也没有必要。IGN的雷耶斯称游戏的遗器系统“令人恼火”,因其升级时提升的属性完全随机。Polygon的迪亚兹认为其角色养成系统比较合理,因为游戏机制鼓励培养多个角色,而非过度投入于一个配队。GamingBolt的辛哈表示游戏升级角色所需材料数量和获取方式之多很容易让玩家不知所措。Rock, Paper, Shotgun的贾森·科尔斯(Jason Coles)、IGN的雷耶斯和GamingBolt的辛哈认为低消费者也能享受该游戏。

《文汇报》记者宣晶认为,《崩坏:星穹铁道》在音乐创作中融合了民族风和电子音乐元素,同时在动画制作中采用了非遗技艺和水墨画风来展现科幻故事,拓展海外玩家对中国文化的理解与认知。《优雅》杂志的刘尚认为,本作以其近乎完美的二次元式美术风格、扣人心弦的剧情设定、独特的游戏世界观和游戏角色的交互性为玩家带来了丰富的游戏体验和独特的审美价值,成为玩家的情感寄托。

\begin{table}[ht]
    \centering
    \caption{获奖}
    \label{tab:table1}
    \fontSimsun\sizeFive[10]
    \newcolumntype{Z}{>{\centering\arraybackslash}X}
    \begin{tabularx}{\textwidth}{Z|Z|Z|Z}
        \hline
        年份                    & 奖项                                              & 类别         & 结果 \\
        \hline
        2022                  & 金摇杆奖                                            & 最受期待游戏     & 提名 \\
        \hline
        \multirow{6}{*}{2023} & 2023世界科幻游戏年度大奖                                  & 最佳人气奖      & 获奖 \\
        \cline{2-4}
        & \multirow{3}{0.25\textwidth}{Google Play年度最佳榜单} & 最佳游戏       & 获奖 \\
        \cline{3-4}
        &                                                 & 最佳剧情游戏     & 获奖 \\
        \cline{3-4}
        &                                                 & 最佳平板游戏     & 获奖 \\
        \cline{2-4}
        & App Store Awards                                & 年度iPhone游戏 & 获奖 \\
        \cline{2-4}
        & 2023年游戏大奖                                       & 最佳手机游戏     & 获奖 \\
        \hline
    \end{tabularx}
\end{table}
