\chapter{绪论}
\vspace{0.5cm}


\section{课题背景}

《崩坏:星穹铁道》设定在一个科学幻想宇宙中\cite{exampleid1},银河中存在着星神,星神是一种高度凝聚的哲学概念化身,是那些践行某种信念并达到命途终点的意识体所成为的存在。星神掌握着改变现实和创造世界的力量,他们可以操控各自命途上的虚数能量。

\subsection{命途}

本作包含18种不同的命途,每个命途都代表了一种信念和力量,其中7种被设定为游戏中的角色职业。每个星神与其命途之间存在一种相互依存的关系。星神可以搬运命途上的能量,但同时也被命途所束缚。一旦命途开启,就无法关闭,即使星神陨落,命途仍然存在。目前已知的18位星神中,有一些已经陨落或失踪。

游戏中的故事情节涉及到不同命途的星神和他们的信念。例如,阿基维利是一个开拓命途的星神,他铺下了星海中的轨道,使得人们可以利用星神的力量进行星际旅行。纳努克则代表了毁灭命途,他认为文明是宇宙的癌症,并散播灾祸的源头到世界各地。还有一些星神代表着不同的命途,如存护命途、欢愉命途、巡猎命途、智识命途、神秘命途、同谐命途等。每个命途都有自己的派系和信徒,这些组织根据对星神和命途的不同理解和信念而形成。派系之间常常结盟或敌对,但多数星神对这些凡人组织并不关心。在游戏中,代表不同命途的派系有毁灭命途的反物质军团、巡猎命途的仙舟联盟和巡海游侠等。此外,还有一些与星神有关的角色,如令使和命途行者。令使是由星神直接赋予力量的凡人,而命途行者则受到星神的感召,走在对应的命途上,并从中获得力量和恩赐。

\subsection{概念}

游戏中还存在一些重要概念,如星核、裂界和星核猎手。星核是被毁灭星神投放到世界各地的世界之癌,会带来灾厄和空间扭曲。裂界是由星核引发的空间扭曲现象,会影响人们并使他们变成怪物。星核猎手则是一个神秘组织,他们在不同地方狩猎星核,认为星核虽然带来了负面影响,但本身是有价值的珍宝。星穹列车是由星神阿基维利制造的列车,在星空轨道上行驶穿越许多世界。它的初始站和末站都叫做裴迦纳,是阿基维利的故乡。

\begin{figure}
    \centering
    \includegraphics[width=100pt]{docs/imgs/HSR}


    \vspace{6pt}
    \fontSimsun\sizeFivel 图中角色为三月七
    \caption{游戏标志}
    \label{fig:figure1}
\end{figure}

\subsection{玩法}

《崩坏:星穹铁道》是一款战略角色扮演游戏,游戏并非开放世界,而是采用箱庭探索式的地图,以星球作为章节区域。世界各地存在多个“界域定锚”,可供玩家用于快速旅行。玩家跟随游戏主线前往不同的地方,还有机关解密及支线任务等内容。世界各地分布着奖励高价值资源的关卡,此类资源可以用于强化角色和购买道具,但需要消耗名为“开拓力”的体力才可进入,体力会随时间推移而缓慢回复。另有名为“模拟宇宙”的Roguelike玩法可获得高级物资。通过完成某些任务或参加特定的限时活动,玩家可解锁一些可玩角色,但大多数角色都需通过名为“跃迁”的抽卡系统获得。游戏设有保底系统,玩家在抽取一定次数后必定获得稀有角色。

战斗模式为回合制,每个队伍可包含四名角色。每个角色都有一种“命途”和元素属性,命途相当于角色职业,分为七种:输出型的智识命途、巡猎命途、毁灭命途,辅助型的虚无命途、存护命途、同谐命途,以及用于治愈的丰饶命途;元素属性则分为七种:物理、火、冰、雷、风、量子、虚数。每个敌人都有一种或多种元素属性作为其“弱点”,使用其弱点属性攻击可降低敌人的“韧性”,韧性耗尽后将对敌人造成“弱点击破”效果,使敌人受到额外伤害并遭受减益,一回合后恢复韧性值。每个角色可在战斗中使用两种技能:“战技”和“终结技”。战技需要消耗“战技点”释放,发动普通攻击可获得战技点;终结技则是需要清空能量才能使用的绝招,普通攻击和战技均可累积能量。