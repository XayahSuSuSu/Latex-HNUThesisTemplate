\chapter{湖南大学}


湖南大学坐落于中国历史文化名城长沙,前临碧波荡漾的湘江,后倚秀如琢玉的岳麓山,素有“千年学府、百年名校”之称。

学校办学起源于公元976年创办的岳麓书院,是中国同址办学时间最长的高等学府,历经千年变迁,弦歌不绝,始终保持文化教育的连续性,是中国高等教育发展的生动缩影和世界高等教育的罕见奇迹。1903年改制为湖南高等学堂,1926年正式定名湖南大学,1937年成为全国16所国立大学之一。新中国成立后,中国共产党主要创始人和早期领导人之一的李达担任第一任校长,毛泽东亲笔题写“湖南大学”校名。2000年,湖南大学与湖南财经学院合并组建成新的湖南大学。改革开放以来,学校先后进入全国重点大学、“211工程”、“985工程”和“世界一流大学”建设高校行列。2020年9月17日,习近平总书记来校考察调研,对学校人才培养、文化传承等给予高度评价,提出岳麓书院是党的实事求是思想路线策源地。

在长期的办学历程中,学校形成了“传道济民、爱国务实、经世致用、兼容并蓄”的教育传统,积淀了以校训“实事求是、敢为人先”、校风“博学、睿思、勤勉、致知”为核心的湖大精神,培育和熏陶了以王夫之、陶澍、魏源、贺长龄、曾国藩、左宗棠、郭嵩焘、谭嗣同、黄兴、蔡锷、杨昌济、毛泽东、何叔衡、蔡和森、邓中夏、李达等为代表的一大批彪炳史册的杰出人才。师生中涌现出42位学部委员和“两院”院士,“惟楚有材,于斯为盛”成为学校人才辈出的生动写照。

学校下设27个学院,学科专业涵盖哲学、经济学、法学、教育学、文学、历史学、理学、工学、管理学、医学、艺术学、交叉学科等12大门类,形成了理科基础坚实、工科实力雄厚、人文社会学科独具特色、生命医学学科兴起、新兴交叉学科活跃的学科布局。拥有本科专业80个,硕士学位授权一级学科37个、博士学位授权一级学科30个。化学、机械工程、电气工程学科进入“世界一流学科”建设行列。

学校现有全日制在校学生37000余人,其中本科生22000余人,研究生15000余人。建有2个教育部基础学科拔尖学生培养计划2.0基地,4个国家级实验教学示范中心,马克思主义学院入选全国重点马克思主义学院。近五年,获国家级教学成果奖16项,首届全国教材建设奖4项,获批“强基计划”招生改革试点,入选国家级一流本科专业建设点54个,国家级一流本科课程70门,获“互联网+”“挑战杯”“创青春”创新创业竞赛国家级金奖9项。学校高度重视国际交流与合作,与加州大学伯克利分校、新加坡国立大学、帝国理工学院等海外130余所高校建立合作关系,招收来自80余个国家和地区的留学生。

学校现有教职工4300余人,其中专任教师2300余人,国家级高层次人才达297人次,其中两院院士全职6人、国家杰出青年科学基金获得者29人次、国家优秀青年科学基金获得者45人次,国家级教学名师5人。拥有国家自然科学基金“创新研究群体”项目6个、国家级教学团队13个,国防科技工业局创新团队1个。

学校科研实力雄厚,科技成果突出。拥有全国(国家)重点实验室6个(含共建)、国家工程技术研究中心2个、国家工程研究中心1个、国家能源研发创新平台1个、教育部集成攻关大平台1个、国家级国际合作基地3个、国防科工局国防重点学科实验室1个、教育部重点实验室和工程研究中心13个、教育部高等学校学科创新引智基地6个、文化和旅游部重点实验室1个。近五年获国家科学技术奖11项,教育部人文社科奖9项。学校坚持产学研相结合,大力促进科技成果转化,建有国家级大学科技园,获批教育部首批高等学校科技成果转化和技术转移基地、国家知识产权示范高校。学校运营管理的国家超级计算长沙中心是第三家国家超级计算中心,天河新一代超级计算机系统算力水平国际先进、国内领先。2023年获批建设国家新一代人工智能公共算力开放创新平台(筹)。

学校校园环境优美,人文气息浓郁。校园占地面积241万平方米,校舍建筑面积135万平方米,典雅厚重的古建筑群与时尚新锐的新建筑体交相辉映,自然风光与人文景观深度融合,被誉为“中国最诗情画意的高校 ”。

“麓山巍巍,湘水泱泱,宏开学府,济济沧沧;承朱张之绪,取欧美之长”,从古代书院到近代学堂再到现代大学,湖南大学坚持追求卓越,始终处在中国高等教育的第一方阵。在新的历史起点上,湖南大学坚持以习近平新时代中国特色社会主义思想为指导,扎根中国大地,矢志一流目标,为把湖南大学早日建成富有历史文化传承的中国特色世界一流大学、培养更多堪当民族复兴大任的建设者和接班人而努力奋进。
